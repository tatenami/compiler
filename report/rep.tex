% ファイル先頭から\begin{document}までの内容(プレアンブル)については,
% 基本的に { } の中を書き換えるだけでよい.
\documentclass[autodetect-engine,dvi=dvipdfmx,ja=standard,
               a4j,11pt]{bxjsarticle}

%%======== プレアンブル ============================================%%
% 用紙設定:指示があれば,適切な余白に設定しなおす
\RequirePackage{geometry}
\geometry{reset,paperwidth=210truemm,paperheight=297truemm}
\geometry{hmargin=25truemm,top=20truemm,bottom=25truemm,footskip=10truemm,headheight=0mm}
%\geometry{showframe} % 本文の"枠"を確認したければ,コメントアウト

% 設定:図の挿入
% http://www.edu.cs.okayama-u.ac.jp/info/tool_guide/tex.html#graphicx
\usepackage{graphicx}

% 設定:ソースコードの挿入
% http://www.edu.cs.okayama-u.ac.jp/info/tool_guide/tex.html#fancyvrb
\usepackage{fancyvrb}
\renewcommand{\theFancyVerbLine}{\texttt{\footnotesize{\arabic{FancyVerbLine}:}}}

%%======== レポートタイトル等 ======================================%%
% ToDo: 提出要領に従って,適切なタイトル・サブタイトルを設定する
\title{情報工学実験C コンパイラ実験最終レポート \\
       |\large{クライアントサーバーモデルで動作する名簿管理プログラムの作成}|}

% ToDo: 自分自身の氏名と学生番号に書き換える
\author{氏名: 寺岡 久騎 (TERAOKA, Hisaki) \\
        学生番号: 09B22433}

% ToDo: レポート課題等の指示に従って適切に書き換える
\date{出題日: 2025年 1月26日 \\
      提出日: 2025年 2月4日 \\
      締切日: 2025年 2月4日 \\}  % 注:最後の\\は不要に見えるが必要.


%%======== 本文 ====================================================%%
\begin{document}
\maketitle
% 目次つきの表紙ページにする場合はコメントを外す
%{\footnotesize \tableofcontents \newpage}

%% 本文は以下に書く.課題に応じて適切な章立てを構成すること.
%% 章=\section,節=\subsection,項=\subsubsection である.

%--------------------------------------------------------------------%
% \section{概要} \label{sec:abstract}


%--------------------------------------------------------------------%
\section{本実験の目的}
% 講義資料No.1を参考

%--------------------------------------------------------------------%
\section{作成した言語の定義}

\subsection{定義した言語で受理されるプログラム}

%--------------------------------------------------------------------%
\section{コード生成の概要}

\subsection{メモリの使用方法}

\subsection{汎用レジスタの使用方法}

\subsection{算術式のコード生成の方法}

% この章にコンパイラのソースプログラムがある場所を記述

%--------------------------------------------------------------------%
\section{最終課題のプログラム及び実行結果}

\section{工夫した点について}

%--------------------------------------------------------------------%
\section{考察}

%--------------------------------------------------------------------%

% Verbatim environment
% プリアンブルで \usepackage{fancyvrb} が必要.
%   - numbers           行番号を表示.left なら左に表示.
%   - xleftmargin       枠の左の余白.行番号表示用に余白を与えたい.
%   - numbersep         行番号と枠の間隙 (gap).デフォルトは 12 pt.
%   - fontsize          フォントサイズ指定
%   - baselinestretch   行間の大きさを比率で指定.デフォルトは 1.0.
% \begin{Verbatim}[numbers=left, xleftmargin=10mm, numbersep=6pt,
%                     fontsize=\small, baselinestretch=0.8]
% #include <stdio.h>

% int main()
% {
%     char s[4] = {'A', 'B', 'C', '\0'};

%     printf("s = %s\n", s);

%     return 0;
% }
% \end{Verbatim}

%--------------------------------------------------------------------%
% 参考文献
%   以下は,書き方の例である.実際に,参考にした書籍等を見て書くこと.
%   本文で引用する際は,\cite{book:algodata}などとすればよい.
% \begin{thebibliography}{99}
%   \bibitem{book:algodata} 平田富雄,アルゴリズムとデータ構造,森北出版,1990.
%   \bibitem{book:label2} 著者名,書名,出版社,発行年.
%   \bibitem{www:label3} WWWページタイトル,URL,アクセス日.
% \end{thebibliography}

%--------------------------------------------------------------------%
%% 本文はここより上に書く(\begin{document}~\end{document}が本文である)
\end{document}
