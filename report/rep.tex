% ファイル先頭から\begin{document}までの内容(プレアンブル)については,
% 基本的に { } の中を書き換えるだけでよい.
\documentclass[autodetect-engine,dvi=dvipdfmx,ja=standard,
               a4j,11pt]{bxjsarticle}

%%======== プレアンブル ============================================%%
% 用紙設定:指示があれば,適切な余白に設定しなおす
\RequirePackage{geometry}
\geometry{reset,paperwidth=210truemm,paperheight=297truemm}
\geometry{hmargin=25truemm,top=20truemm,bottom=25truemm,footskip=10truemm,headheight=0mm}
%\geometry{showframe} % 本文の"枠"を確認したければ,コメントアウト

% 設定:図の挿入
% http://www.edu.cs.okayama-u.ac.jp/info/tool_guide/tex.html#graphicx
\usepackage{graphicx}
\usepackage{caption}
\usepackage{amsmath}

% 設定:ソースコードの挿入
% http://www.edu.cs.okayama-u.ac.jp/info/tool_guide/tex.html#fancyvrb
\usepackage{fancyvrb}
\renewcommand{\theFancyVerbLine}{\texttt{\footnotesize{\arabic{FancyVerbLine}:}}}

%%======== レポートタイトル等 ======================================%%
% ToDo: 提出要領に従って,適切なタイトル・サブタイトルを設定する
\title{情報工学実験C コンパイラ実験最終レポート}

% ToDo: 自分自身の氏名と学生番号に書き換える
\author{氏名: 寺岡 久騎 (TERAOKA, Hisaki) \\
        学生番号: 09B22433}

% ToDo: レポート課題等の指示に従って適切に書き換える
\date{出題日: 2025年 1月26日 \\
      提出日: 2025年 2月3日 \\
      締切日: 2025年 2月4日 \\}  % 注:最後の\\は不要に見えるが必要.


%%======== 本文 ====================================================%%
\begin{document}
\maketitle
% 目次つきの表紙ページにする場合はコメントを外す
%{\footnotesize \tableofcontents \newpage}

%% 本文は以下に書く.課題に応じて適切な章立てを構成すること.
%% 章=\section,節=\subsection,項=\subsubsection である.

%--------------------------------------------------------------------%
% \section{概要} \label{sec:abstract}


%--------------------------------------------------------------------%
\section{本実験の目的}
% 講義資料No.1を参考
本実験の目的は,これまで学んできたC言語についての知識を再確認するとともに,
それを使って,
\begin{itemize}
  \item ソフトウェア全体の仕様の決定
  \item プログラムで利用するデータ構造,アルゴリズムの考案と実装
  \item 動作のテスト,デバッグの作業
\end{itemize}
これら一連の作業を通して大規模なプログラムの作成の経験をすること,そして
lex,yaccといったプログラムジェネレータを使用した
コンパイラプログラムの作成を通して,コード解析やファイルシステムに
利用される木構造の取り扱い,ソースコードとアセンブリ言語との対応について
の理解と習熟を深めることである.

今回作成したコンパイラプログラムの実行ファイルは,
演習室のPCの \\
\verb|/home/users/ecs/09B22433/exp-c/compiler/lang|ディレクトリに
\verb|lang2|という名前で保存している.

%--------------------------------------------------------------------%
\section{作成した言語の定義}
以下に作成した言語の定義として,文法規則を記述したyaccのルールを示す.
言語は,最終課題1--5が実行できる実装となっている.

\begin{Verbatim}[numbers=left, xleftmargin=10mm, numbersep=6pt, frame=single,
                    fontsize=\small, baselinestretch=0.8]
%union {
  struct node *np;
  char* sp;
  int   ival;
}

%type <np> program declarations decl_statement decl_part
           statements statement 
           assignment_stmt assignment loop_stmt cond_stmt 
           while_stmt if_stmt else_stmt elif_stmt for_stmt
           expressions expression 
           condition array array_index term factor unary_factor 
           comp_op unary_op bit_op var idents

%type <ival> add_op mul_op

%token DEFINE ASSIGN ARRAY_DEF 
        L_BRACKET R_BRACKET L_PARAN R_PARAN L_BRACE R_BRACE
        SEMIC COMMA ADD SUB MUL DIV REM INCREM DECREM EQ NE LT GT LTE GTE
        AND OR XOR NOT L_SHIFT R_SHIFT
        FOR WHILE IF ELSE

%token <sp> IDENT
%token <ival> NUMBER

%define parse.error verbose

%%

program
  : declarations statements 
  | declarations           
;

declarations
  : decl_statement declarations 
  | decl_statement             
;

array_index
  : L_BRACKET expression R_BRACKET 
  | L_BRACKET expression R_BRACKET L_BRACKET expression R_BRACKET 
;

array
  : IDENT array_index 
;

decl_part
  : DEFINE idents 
  | ARRAY_DEF array 
;

decl_statement
  : decl_part SEMIC   
;

statements
  : statement statements 
  | statement          
;

statement
  : assignment_stmt
  | loop_stmt       
  | cond_stmt     
;

assignment
  : IDENT ASSIGN expression 
  | array ASSIGN expression  
  | unary_factor 
;

assignment_stmt
  : assignment SEMIC 
;

expressions
  : expression COMMA expressions 
  | expression                   
;

expression
  : expression add_op term 
  | term                   
;

term
  : term mul_op factor
  | term bit_op factor 
  | factor             
;

unary_factor
  : IDENT unary_op 
  | unary_op IDENT 
;

factor
  : var                         
  | L_PARAN expression R_PARAN                 
  | unary_factor                
  | NOT IDENT
; 

add_op
  : ADD 
  | SUB 
;

mul_op
  : MUL 
  | DIV 
  | REM 

unary_op
  : INCREM 
  | DECREM 
;

bit_op
  : AND     
  | OR      
  | XOR     
  | L_SHIFT 
  | R_SHIFT 
;

var
  : IDENT                       
  | NUMBER                      
  | IDENT array_index
;

loop_stmt
  : while_stmt
  | for_stmt
;

while_stmt
  : WHILE L_PARAN condition R_PARAN L_BRACE statements R_BRACE 
;

for_stmt
  : FOR L_PARAN assignment SEMIC condition SEMIC assignment R_PARAN 
  L_BRACE statements R_BRACE 
;

cond_stmt
  : if_stmt   
  | elif_stmt 
;

if_stmt
  : IF L_PARAN condition R_PARAN L_BRACE statements 
;

else_stmt
  : ELSE L_BRACE statements R_BRACE
;

elif_stmt
  : if_stmt else_stmt 
;

condition
  : expression comp_op expression 
;

comp_op
  : EQ  
  | NE  
  | LT  
  | GT  
  | LTE 
  | GTE 
;

idents
  : IDENT COMMA idents 
  | IDENT
;

%%
\end{Verbatim}

\subsection{定義した言語で受理されるプログラム}

\subsubsection{全体の構造}
作成した言語では,プログラム全体はまず
\begin{itemize}
  \item 変数宣言部
  \item 処理文集合
\end{itemize}
からなる仕様となっており,使用される変数はプログラムの冒頭で
全て宣言され,その後に
変数が使用される様々な処理文の集合が記述される.従って,文集合中に変数宣言は行われない.

\subsubsection{変数の宣言}
プログラムの変数宣言部では,4バイトの整数値を格納できる通常の変数と
,この整数値を複数格納できる配列(1次元,2次元)が記述される.
yaccの規則から,プログラムにおいて変数の宣言として以下の記述ができる.
ここで,\verb|<識別子名>|は変数名であり,先頭がアルファベットの英数字列である.
\begin{description}
  \item[整数変数:] \verb|define <識別子名>;|
  \item[整数変数(複数):] \verb|define <識別子名>, ...;|
  \item[配列(1次元):] \verb|array <識別子名>[自然数];|
  \item[配列(2次元):] \verb|array <識別子名>[自然数][自然数];|
\end{description}

\subsubsection{処理文}
文集合では,C言語と同様の
\begin{itemize}
  \item 代入文
  \item ループ文(\verb|while|,\verb|for|)
  \item 条件分岐文(\verb|if|,\verb|if-else|)
  \item 算術式
  \item 条件式
\end{itemize}
による処理が記述できる.
式については,
条件式では比較演算子が,
そして算術式では算術演算子が利用できる.
以下,実装した各種演算子についての概説を行う.

\paragraph*{算術演算子}
以下,それぞれ実装した
算術式における演算子を示す.\verb|a|及び\verb|b|は被演算子であり,
整数の即値,変数,算術式(の演算結果)が対象となる.また,
演算対象とその演算結果は4バイトの符号付き整数である.
\begin{itemize}
  \item \texttt{a + b}: \ \verb|a|と\verb|b|の値を加算する
  \item \texttt{a - b}: \ \verb|a|の値から\verb|b|の値を減する
  \item \texttt{a * b}: \ \verb|a|と\verb|b|の値を乗算する
  \item \texttt{a / b}: \ \verb|a|の値を\verb|b|で除する
  \item \texttt{a \% b}: \ \verb|a|の値を\verb|b|の値で割った時の剰余を計算する  
  \item \texttt{a++・(a--)}: \ \verb|a|の値を1だけ増加(減少)する.ただし,この処理は\verb|a|を被演算子や対象とする代入・算術処理の後に行われる.     
  \item \texttt{++a・(--a)}: \ \verb|a|の値を1だけ増加(減少)する.ただし,この処理は\verb|a|を被演算子や対象とする代入・算術処理の前に行われる.     
\end{itemize}

\paragraph*{比較演算子}
比較式において,その式の結果は
条件が成立する場合は1,成立しない場合は0となる(詳細は第\ref{sec:reg}節に示す).
以下,それぞれ実装した演算子を示す.
\verb|a|及び\verb|b|は被演算子であり,
整数の即値,変数,算術式(の演算結果)が対象となる.
\begin{itemize}
  \item \texttt{a == b}: \ \verb|a|と\verb|b|の値が一致しているを比較する
  \item \texttt{a != b}: \ \verb|a|と\verb|b|の値を一致していないかを比較する
  \item \texttt{a < b}: \ \verb|a|の値より\verb|b|の値が大きいかを比較する
  \item \texttt{a > b}: \ \verb|a|の値が\verb|b|の値より大きいかを比較する
  \item \texttt{a <= b}: \ \verb|a|の値が\verb|b|の値以下であるかを比較する     
  \item \texttt{a >= b}: \ \verb|a|の値が\verb|b|の値以上であるかを比較する     
\end{itemize}

\subsubsection{受理されるプログラムの例}
以上の定義した言語の使用から,実際に作成したコンパイラに受理されるプログラムの例を示す.
ただし,プログラム自体は意味のある処理を行うものではない.

\begin{Verbatim}[numbers=left, xleftmargin=10mm, numbersep=6pt, frame=single,
  fontsize=\small, baselinestretch=0.8]
define i;
define j, k;
array  m1[10];
araay  m2[3][3];
i = 1;
j = i + 1;
k = i * j;

while (i <= 100) {
  i = i + 2;
  if (i % 2 == 1) {
    j = i / j - 1;
  }
  else {
    --k;
  }
}

for (i = 1; k != 10; k++) {
  if (i > j) {
    j = k - i--;
  }
}
\end{Verbatim}


%--------------------------------------------------------------------%
\section{コード生成の概要}
作成したコンパイラプログラムでは,以下の一連の流れで処理を行う.
\begin{enumerate}
  \item 入力された文字列について,
  lexファイルで定義した文字列でトークンに該当するものをトークンへ置換,そして全体をトークン列に置換する
  \item プログラムのトークン列に対してyaccで記述した文法規則に従って還元を行い,
  この時に生成規則のアクション部に記述したC言語の処理により,各トークンをノードとし,
  プログラム全体のトークンからコード生成に必要な要素を残して抽象構文木を
  作成する.
  \item 構成された抽象構文木をたどり,ノードの構成などから
  プログラム実行に必要な処理を出力対象のファイルへmapsで実行可能なアセンブリ言語の命令を記述する.
\end{enumerate}

入力されたプログラムのコード生成(ファイルの作成)は,上記の項目3の抽象構文木の解析・命令の記述,及び
この前後に行う以下の処理により行われる.
\begin{itemize}
  \item (項目3実行前) 抽象構文木からプログラムで使用する変数の記号表の作成
  \item (項目3実行前) mapsプログラム実行に必要なテキストセグメント宣言等の初期化処理の記述
  \item (項目3実行後) mapsプログラム正常終了に必要な処理,データセグメントの記述
\end{itemize}

mapsで実行可能なコードの生成について,上記の
抽象構文木を用いた解析に基づく命令の記述と前後処理を踏まえて,以下の項目に分けて解説を行う.
\begin{itemize}
  \item メモリ領域の使用方法
  \item mapsの汎用レジスタの使用方法
  \item 算術式のコード生成の方法
\end{itemize}

\subsection{メモリの使用方法}
プログラムにおいて,メモリ領域は主に用途により
\begin{itemize}
  \item 実行する命令が配置されるテキストセグメント
  \item プログラムで使用される変数が格納されるデータセグメント
  \item 算術式等の計算結果の一時保存に使用するスタック領域
\end{itemize}
に分けられる.以下,上記の用途によるメモリの利用について概説を行う.

\subsubsection{テキストセグメントの利用}
テキストセグメントについて,mapsでプログラムを正常終了するための初期化処理
をコード生成の直前に,以下の処理を記述する
\begin{itemize}
  \item グローバルポインタと後述するスタックポインタの値の設定
  \item テキストセグメントの配置を制御するアセンブラ命令(\verb|.text|)
  \item グローバルポインター・スタックポインタのレジスタへの格納
  \item プログラム終了前に行う処理
\end{itemize}
そして,その直後にソースコードで実行される
テキストセグメントの配置を制御するアセンブラ命令を記述する.
このセグメントのアドレスは上記の初期化処理の命令を考慮し,\verb|0x00001000|
として設定した.
即ち,ソースコードから生成した命令が配置されるのは,メモリにおいて
\verb|0x00001000|から後述するデータセグメントまでの領域としている.

\subsubsection{データセグメントの利用}
データセグメントについては,メモリ領域の\verb|0x10004000|領域以降の
アドレスを利用している.実際に生成するコードではこの
アドレスを保持した汎用レジスタを利用と,
先述の変数の記号表で保持している変数のデータセグメントの先頭からのオフセット
を利用して,この領域に各変数を格納する.
変数の記号表は,ソースコードに宣言された順にサイズと
オフセットをそれぞれ計算しており,この際始めに宣言された変数はデータセグメントの先頭アドレスを
以降の変数は直前に宣言された変数のオフセットを格納する領域のアドレスとしているため,
データセグメントは,先頭アドレスから順に変数を並んで格納するように使用される.

以下,例として通常の変数(4バイト)を3つ使用する際のデータセグメントにあたるメモリ領域の使用状態を示す.
順に\verb|a|,\verb|b|,\verb|c|と宣言したとする.

\begin{itemize}
  \item\texttt{0x10004000} -- \texttt{0x10004003}: \verb|a| (4Byte) 
  \item\texttt{0x10004004} -- \texttt{0x10004007}: \verb|b| (4Byte) 
  \item\texttt{0x10004008} -- \texttt{0x1000400b}: \verb|c| (4Byte) 
\end{itemize}

\subsubsection{スタック領域の利用}
プログラムの算術式等の処理で一時的なデータの保存に利用するスタック
領域の先頭アドレスは\verb|0x7ffffffc|としている.
先述の2つの領域とは異なり,データをスタックに積む処理を行う場合は
このアドレスよりも小さいアドレスへ変数のデータを格納する.

以下,例として通常の変数(4バイト)を3を一時的にスタックへ積む処理を行った際のメモリ領域の使用状態を示す.
順に変数として\verb|a|,\verb|b|,\verb|c|を順に格納したとする.

\begin{itemize}
  \item\texttt{0x7ffffff4} -- \texttt{0x7ffffff7}: \verb|c| (4Byte) 
  \item\texttt{0x7ffffff8} -- \texttt{0x7ffffffb}: \verb|b| (4Byte) 
  \item\texttt{0x7ffffffc} -- \texttt{0x7fffffff}: \verb|a| (4Byte) 
\end{itemize}
仮に\verb|c|がスタックから取り出すことに該当する処理が行われた場合,次にスタックへ積まれるデータは
\verb|c|のデータがあった領域に格納される.

\subsection{レジスタの使用方法} \label{sec:reg}
プログラムコードでは様々な処理で一時的にデータを保持する必要があるため,
mipsで使用できる汎用レジスタを利用した.
レジスタについてはコンパイラプログラムのコード生成の処理の複雑化を避けるために,
その一部に固定の役割を与えて使用した.
表\ref{tab:register-tab}に使用した汎用レジスタとその役割・用途について示す.

\begin{table}[htbp]
  \centering
  \begin{tabular}{|c|c|}
    \hline
      レジスタ & 役割・用途 \\
    \hline
    \hline
      \verb|$t0| & データセグメントの先頭アドレス(\verb|0x10004000|)を保持する \\
    \hline
      \verb|$t1| & 比較演算の演算結果の値の格納先 \\
    \hline
      \verb|$t2| & 算術式・条件式の第1オペランド(左側)の値の格納先 \\
    \hline
      \verb|$t3| & 算術式・条件式の第2オペランド(右側)の値の格納先 \\
    \hline
      \verb|$v0| & 算術式の計算結果の一時的な保持 \\
    \hline
  \end{tabular}
  \caption{汎用レジスタの役割・用途}
  \label{tab:register-tab}
\end{table}

\verb|$t0|レジスタはプログラム実行時の始めにデータセグメントの先頭アドレスを
格納し,以降はデータセグメントの領域に格納されている変数に対してロードやストアを
行う際のアドレスとして利用する.そのため,プログラムの処理で値を変更しない.

\verb|$t1|レジスタは条件式の演算結果の格納先としている.
これにより,ループ文や分岐処理文においてジャンプ命令・比較演算の
コードの生成がこれレジスタを必ず参照するようにすることで
条件式の生成処理に依存せず,コンパイラプログラム全体の処理の拡張・修正の
実装が容易に行えるようにした.

\verb|$t2|及び\verb|$t3|は演算処理においてオペランドの値の格納先として
利用した.これにより,オペランドに更に算術式が連なる複雑な式の再帰的な生成処理において,
使用するレジスタの削減と算術式関連のプログラムの複雑化を避けた.

\subsection{算術式のコード生成の方法} \label{sec:expresssion}
算術式は,抽象構文木において演算子を親ノードとし,
演算対象となる2つのオペランドのノードが子ノードとなる2分木の構造となっている.
そのため,演算処理のコードの生成にあたってはオペランドの値を
第\ref{sec:reg}節に示したレジスタに格納した上で,それを対象として各計算の
命令を記述する.計算結果は\verb|$v0|に格納し,その値をスタックへプッシュする.

演算子の対象となるオペランドのノードには以下のいずれかが該当する.
\begin{itemize}
  \item 整数の即値のノード
  \item 変数のノード(通常の変数・配列の1要素)
  \item 算術式の演算子のノード
\end{itemize}
オペランドとして更に算術式が連なっている場合があり,この場合ではより深い側の演算は
計算の優先順序が高いため,
正しい計算順序でコードの生成を行うには,木の
より深い側の演算から行う必要がある.
これと先述の算術式に関する値の格納に使用するレジスタの扱いを考慮すると,
算術式のコード生成の処理は以下の流れで行われるよう実装した.

\begin{enumerate}
  \item 親にあたる演算子のノードの2つの子ノードが算術演算子であるかを確認する
  \begin{itemize}
    \item 子ノードが算術演算子の場合,1の処理からはじまる一連の処理を,このノードを親ノードとして再帰的に呼び出す
    (第1オペランド,第2オペランドの順でこの処理を行う)
  \end{itemize}
  \item 第2オペランドのノードが算術演算子であるかを確認し,該当するかどうかで以下処理を行う
  \begin{description}
    \item[算術演算子の場合:] 演算結果がスタックに積まれているため,そのデータをスタックからポップして\verb|$t3|レジスタ 
    へロードする
    \item[算術演算子でない場合:] 該当の数値や変数の値を\verb|$t3|レジスタに格納する   
  \end{description}
  \item 第1オペランドに対して2の処理を行う(格納先のレジスタは\verb|$t2|となる)
  \item オペランドの値が格納された\verb|$t2|レジスタ,\verb|$t3|レジスタを利用して
  演算子に対応した命令を記述する.ここで計算結果の格納先は\verb|$v0|レジスタとする
  \item 計算結果が格納された\verb|$v0|レジスタの値をスタックへプッシュする
\end{enumerate}
以上の処理により,算術式に対してオペランドに算術式が続くような形であっても,再帰的な処理によって
深さ優先探索の要領で優先順序が高い計算を実行する命令が先に記述され,計算結果が
スタックに保存された状態になり,その後にこの計算結果を取り出して使用
する親側の算術式が記述される.
これによって算術演算が連なる式であっても正しい順序で計算が行われるコードが生成される.

\section{工夫した点について}

コンパイラプログラムの作成において,
生成されるコードにおける最適化・実行サイクル数の削減
に関する以下の工夫・改良を行った.
\begin{itemize}
  \item 式におけるオペランドのデータの遅延ロードを考慮した\verb|nop|命令の削減
  \item 算術式・条件式におけるオペランドの種類による最適なデータ取得処理の選択
\end{itemize}

\subsection{オペランドのロード処理時\texttt{nop}命令の削減}
算術式・条件式においては,第\ref{sec:expresssion}節で示したように
演算対象である2つのオペランドのデータをレジスタへ格納を行う.
対象となるデータの内,変数(配列の要素)とさらに連なる算術式の結果を
レジスタに格納する場合,ロード命令(\verb|lw|)を使用する.
mapsにおいてはこのロードが完了してレジスタにデータが格納されるには1サイクルの遅延がかかり,
直後にそのデータを用いる処理が正常に動作しないため,その対策として通常は以下の手順の様にロード命令直後に
\verb|nop|命令を追加で記述する必要がある.

\begin{enumerate}
  \item 第2オペランドのデータのロード(\verb|$t3|レジスタ)
  \item \verb|nop|命令(遅延ロード対策)
  \item 第1オペランドのデータのロード(\verb|$t2|レジスタ)
  \item \verb|nop|命令(遅延ロード対策)
\end{enumerate}

今回の実装では,オペランドのデータのロードは必ず第2オペランドのデータを格納する処理の後に
第1オペランドのデータを格納する処理が記述される.そのため,
第2オペランドの処理の後に対象のレジスタ(\verb|$t3|)を使用する処理が記述されないため,
第ロード命令によりレジスタにデータを格納する場合,
直後に\verb|nop|命令が不要となるため,以下の処理手順の様に
オペランドのデータ格納時に対象が第2オペランドである場合は\verb|nop|命令を記述しない
ようにすることで,サイクル数の削減を行った.

\begin{enumerate}
  \item 第2オペランドのデータのロード(\verb|$t3|レジスタ)
  \item 第1オペランドのデータのロード(\verb|$t2|レジスタ)
  \item \verb|nop|命令(遅延ロード対策)
\end{enumerate}

\subsection{算術式・条件式におけるデータの取得時の最適なデータ取得処理の選択}
算術式のオペランドとして算術式が連なるような長い式では,子ノードにあたる
式の計算結果を親にあたる演算で用いるため,一時的にその結果を保存しておく必要がある.
この場合,オペランドが算術式かそうでないかに関わらず,
データをスタックへプッシュし,演算の際に2つのオペランドをポップする処理を
,抽象構文木の演算子のノードから子ノードをたどって再帰的に行うことで
オペランドのデータあるいは計算結果を利用して演算処理を行うことができる.
しかし,この処理では本来レジスタに格納するだけで使用できる整数の即値と変数のデータに
対してスタックにプッシュしてポップする操作を行っており,余分にサイクル数がかかっていると考えられる.

そのため,単に再帰的処理の中で全てのデータに対してスタックを使用するのではなく,算術式にあたる構文木に対して,
より深い優先すべき計算から順に算術演算子に該当するノードを対象として
コード生成を行い,オペランドが算術式かそうでないかで
スタックからデータを取り出すか,単にレジスタに値を格納するかを分岐させることで,
常に必要なデータの格納処理を選択する処理を実現し,サイクル数の削減を図った.


\subsection{最適化による実行サイクル数の変化}
今回のコンパイラプログラムでは,以上2つの最適化の工夫を施した上で実装した.
この工夫により生成した最終課題のコードのサイクル数ついて,
\begin{itemize}
  \item オペランドを種類に依らず必ずスタックへプッシュし,都度ポップする
  \item レジスタへのロードの直後に必ず\verb|nop|命令を記述する
\end{itemize}
という最適化を施さない処理で生成した場合と比較した結果を表\ref{tab:cycle-tab}に示す.

いずれのプログラムでも約40\%程実行サイクル数が削減された.
課題の内,変数の宣言と値の初期化処理を除いた処理において,
算術式に即値を用いる処理の割合が比較的高い課題1,2,3では
やや削減率が高いことからも,スタックを用いずにレジスタへへ直接データを格納
する処理にしたことによるサイクル数の減少の効果が大きいことが考えられる.

\begin{table}[htbp]
  \centering
  \begin{tabular}{|c|r|r|r|}
    \hline
      課題 & 最適化なし & 最適化あり & サイクル数の削減率(\verb|%|)\\
    \hline
    \hline
      課題1 & 2325 & 1231 & 47.06\\
    \hline
      課題2 & 1245 & 681 & 45.31\\
    \hline
      課題3 & 15341 & 8269 & 46.10\\
    \hline
      課題4 & 2125655 & 1212350 & 42.97\\
    \hline
      課題5 & 7582 & 4450 & 41.31 \\
    \hline
  \end{tabular}
  \caption{最適化によるプログラムの合計実行サイクル数}
  \label{tab:cycle-tab}
\end{table}


%--------------------------------------------------------------------%
\section{最終課題のプログラム及び実行結果}
定義した言語及びそれを対象としたコンパイラプログラムに
より実行が可能な課題として,以下の最終課題に取り組んだ.
\begin{description}
  \item[課題1] 1から10までの数の和の計算
  \item[課題2] 5の階乗の計算
  \item[課題3] FizzBuzz問題
  \item[課題4] エラトステネスのふるいを用いた素数の探索
  \item[課題5] 2つの$2 \times 2$行列の積の計算 
\end{description}

これら最終課題を解くにあたって記述したプログラムのコードと,
それを作成したコンパイラによりmapsアセンブリ言語へ変換したコードについて,
\begin{itemize}
  \item シミュレーション上での実行によるサイクル数・命令数の出力結果
  \item コンパイル時に出力する,プログラムで使用した変数の情報
  \item 実行後の変数に関するメモリ領域の出力結果
\end{itemize}
を示す.

\subsection{課題1 1から10までの数の和の計算}

\paragraph*{ソースコード}

\begin{Verbatim}[numbers=left, xleftmargin=10mm, numbersep=6pt, frame=single,
                    fontsize=\small, baselinestretch=0.8]
define i;
define sum;
sum = 0;
i = 1;
while(i < 11) {
  sum = sum + i;
  i = i + 1;
}
\end{Verbatim}

\subsubsection{実行・出力結果}
実行の結果,合計$1231$サイクル(ステップ)で処理が終了した.
プログラムでは,\verb|sum|は和を格納する変数として,
変数\verb|i|は加算する数として使用される.\verb|i|は
\verb|while|文中で1ずつ加算され,値が$11$となった
時点で処理が終了する.
実行後の変数のメモリから,\verb|i|は$b_{(16)} = 11$,
\verb|sum|が$37_{(16)} = 55$となっており,
正しく1から10までの和が計算されたことが確認できた.

\paragraph*{実行時の命令・サイクル数}

\begin{Verbatim}[numbers=none, frame=single,
fontsize=\small, baselinestretch=0.8]
*** multicycle statistics ***
*** total inst. count:          303
*** total cycle count:         1231
***               IPC:   0.246 (inst/cycle)
***               CPI:   4.061 (cycle/inst)  
\end{Verbatim}

\paragraph*{使用した変数の情報}
\begin{Verbatim}[numbers=none, frame=single,
  fontsize=\small, baselinestretch=0.8]
# ------ [ Symbols ] ------
# 	symbol_0 	size: 4 	offset:     0(0) 	[i]
# 	symbol_1 	size: 4 	offset:   0x4(4) 	[sum]
\end{Verbatim}

\paragraph*{実行後の変数メモリ}
\begin{Verbatim}[numbers=none, frame=single,
  fontsize=\small, baselinestretch=0.8]
# tag=256 index=4
10004000: 0000000b 00000037 00000000 00000000
\end{Verbatim}

\subsection{課題2 5の階乗の計算}
\paragraph*{ソースコード}

\begin{Verbatim}[numbers=left, xleftmargin=10mm, numbersep=6pt, frame=single,
  fontsize=\small, baselinestretch=0.8]
define i;
define fact;
fact = 1;
i = 1;
while(i < 6) {
  fact = fact * i;
  i = i + 1;
}
\end{Verbatim}

\subsubsection{実行・出力結果}
実行の結果,合計$681$サイクル(ステップ)で処理が終了した.
プログラムでは,\verb|fact|は階乗の値を格納する変数として,
変数\verb|i|は乗算する数として使用される.\verb|i|は
\verb|while|文中で1ずつ加算され,値が$6$となった
時点で処理が終了する.
実行後の変数のメモリから,\verb|i|は$6$,
\verb|sum|が$78_{(16)} = 120$となっており,
1から5が乗算され,正しく階乗の計算が行われたことが確認できた.

\paragraph*{実行時の命令・サイクル数}
\begin{Verbatim}[numbers=none, frame=single,
fontsize=\small, baselinestretch=0.8]
*** multicycle statistics ***
*** total inst. count:          168
*** total cycle count:          681
***               IPC:   0.247 (inst/cycle)
***               CPI:   4.052 (cycle/inst)
\end{Verbatim}

\paragraph*{使用した変数の情報}
\begin{Verbatim}[numbers=none, frame=single,
  fontsize=\small, baselinestretch=0.8]
# ------ [ Symbols ] ------
# 	symbol_0 	size: 4 	offset:     0(0) 	[i]
# 	symbol_1 	size: 4 	offset:   0x4(4) 	[fact]  
\end{Verbatim}

\paragraph*{実行後の変数メモリ}
\begin{Verbatim}[numbers=none, frame=single,
  fontsize=\small, baselinestretch=0.8]
# tag=256 index=4
10004000: 00000006 00000078 00000000 00000000
\end{Verbatim}

\subsection{課題3 FizzBuzz問題}
本課題は,基本言語仕様を拡張して実装した剰余演算子を
用いて記述した.

\paragraph*{ソースコード}
\begin{Verbatim}[numbers=left, xleftmargin=10mm, numbersep=6pt, frame=single,
  fontsize=\small, baselinestretch=0.8]
define fizz;
define buzz;
define fizzbuzz;
define others;
define i;
fizz = 0;
buzz = 0;
fizzbuzz = 0;
others = 0;
i = 1;
while(i < 31){
  if (i % 15 == 0){
    fizzbuzz = fizzbuzz + 1;
  } else {
    if (i % 3 == 0){
      fizz = fizz + 1;
    } else {
      if (i % 5 == 0){
        buzz = buzz + 1;
      } else {
        others = others + 1;
      }
    }
  }
  i = i + 1;
}  
\end{Verbatim}

\subsubsection{実行・出力結果}
実行の結果,合計$8269$サイクル(ステップ)で処理が終了した.
プログラムにおいてFizzBuzz問題の対象は1から30であり,
15の倍数は2個存在するため,FizzBuzzの該当数は2となる.
このことから,Fizzは$30 / 3 - 2$で8,Buzzは$30 / 5 - 2$
で4となる.実行後の変数メモリでは,
\verb|Fizz|が8,\verb|Buzz|が4,\verb|FizzBuzz|が2,
となっており,対象の数として\verb|while|文の処理で1加算される
\verb|i|が$\verb|1f|_{(16)} = 31$となっており,
正しく処理が行われたことが確認できた.

\paragraph*{実行時の命令・サイクル数}
\begin{Verbatim}[numbers=none, frame=single,
fontsize=\small, baselinestretch=0.8]
*** multicycle statistics ***
*** total inst. count:         2092
*** total cycle count:         8269
***               IPC:   0.253 (inst/cycle)
***               CPI:   3.953 (cycle/inst)
\end{Verbatim}

\paragraph*{使用した変数の情報}
\begin{Verbatim}[numbers=none, frame=single,
  fontsize=\small, baselinestretch=0.8]
# ------ [ Symbols ] ------
# 	symbol_0 	size: 4 	offset:     0(0) 	[fizz]
# 	symbol_1 	size: 4 	offset:   0x4(4) 	[buzz]
# 	symbol_2 	size: 4 	offset:   0x8(8) 	[fizzbuzz]
# 	symbol_3 	size: 4 	offset:   0xc(12) 	[others]
# 	symbol_4 	size: 4 	offset:  0x10(16) 	[i]  
\end{Verbatim}

\paragraph*{実行後の変数メモリ}
\begin{Verbatim}[numbers=none, frame=single,
  fontsize=\small, baselinestretch=0.8]
# tag=256 index=4
10004000: 00000008 00000004 00000002 00000010
10004010: 0000001f 00000000 00000000 00000000
\end{Verbatim}

\subsection{課題4 エラトステネスのふるいを用いた素数の探索}
本課題では,1から1000までの数を対象となっている.
\paragraph*{ソースコード}

\begin{Verbatim}[numbers=left, xleftmargin=10mm, numbersep=6pt, frame=single,
  fontsize=\small, baselinestretch=0.8]
define N;
define i;
define j;
define k;
array a[1001];
N = 1000;
i = 1;
while (i <= N) {
  a[i] = 1;
  i = i + 1;
}
i = 2;
while(i <= N/2) {
  j = 2;
  while(j <= N/i){
    k = i * j;
    a[k] = 0;
    j = j + 1;
  }
  i = i + 1;
}
\end{Verbatim}

\subsubsection{実行・出力結果}
このプログラムでは,配列の\verb|a|の
素数に該当する数の要素が1,そうでない合成数の要素が
0となる.
配列\verb|a|はアドレス\verb|0x10004010|から
はじまる領域に格納され,各要素が4バイトとなっており,
変数のメモリにおいて,最初の4バイトが添字として0に
該当する要素である.このことから,
実行後の変数メモリで0,1番目の要素を除いて
2,3,5,7,11,13,17,19,23番目の値が1となっており,
正しく素数の探索処理が行えたことが確認できる.

\paragraph*{実行時の命令・サイクル数}
\begin{Verbatim}[numbers=none, frame=single,
fontsize=\small, baselinestretch=0.8]
*** multicycle statistics ***
*** total inst. count:       294858
*** total cycle count:      1212350
***               IPC:   0.243 (inst/cycle)
***               CPI:   4.107 (cycle/inst)
\end{Verbatim}

\paragraph*{使用した変数の情報}
\begin{Verbatim}[numbers=none, frame=single,
  fontsize=\small, baselinestretch=0.8]
# ------ [ Symbols ] ------
# 	symbol_0 	size: 4 	offset:     0(0) 	[N]
# 	symbol_1 	size: 4 	offset:   0x4(4) 	[i]
# 	symbol_2 	size: 4 	offset:   0x8(8) 	[j]
# 	symbol_3 	size: 4 	offset:   0xc(12) 	[k]
# 	symbol_4 	size: 4004 	offset:  0x10(16) 	[a]  
\end{Verbatim}

\paragraph*{実行後の変数メモリ}
出力されたメモリ領域が大きいため,省略した冒頭の一部を示す.

\begin{Verbatim}[numbers=none, frame=single,
  fontsize=\small, baselinestretch=0.8]
  # tag=256 index=4
10004000: 000003e8 000001f5 00000003 000003e8
10004010: 00000000 00000001 00000001 00000001
10004020: 00000000 00000001 00000000 00000001
10004030: 00000000 00000000 00000000 00000001
10004040: 00000000 00000001 00000000 00000000
10004050: 00000000 00000001 00000000 00000001
10004060: 00000000 00000000 00000000 00000001
\end{Verbatim}

\subsection{課題5 2つの$2 \times 2$行列の積の計算}
\paragraph*{ソースコード}

\begin{Verbatim}[numbers=left, xleftmargin=10mm, numbersep=6pt, frame=single,
  fontsize=\small, baselinestretch=0.8]
array matrix1[2][2];
array matrix2[2][2];
array matrix3[2][2];
define i, j, k;
matrix1[0][0] = 1;
matrix1[0][1] = 2;
matrix1[1][0] = 3;
matrix1[1][1] = 4;
matrix2[0][0] = 5;
matrix2[0][1] = 6;
matrix2[1][0] = 7;
matrix2[1][1] = 8;
for(i=0;i<2;i++){
  for(j=0;j<2;j++){
   matrix3[i][j] = 0;
  }
}
for(i=0;i<2;i++){
  for(j=0;j<2;j++){
    for(k=0;k<2;k++){
      matrix3[i][j] = matrix3[i][j] + matrix1[i][k] * matrix2[k][j];
    }
  }
}
\end{Verbatim}

\subsubsection{実行・出力結果}
実装した2次元配列では,
1次元の配列が順に行数文メモリに格納される仕様であり,
例として2次元配列\verb|a[3][2]|が格納されるメモリ領域では,
先頭から\verb|a[0][0]|,\verb|a[0][1]|,\verb|a[1][0]|,\verb|a[1][1]|...
の順に要素が並ぶ.

このプログラムで計算する行列の積は,
\begin{equation*}
  \begin{pmatrix}
    1 & 2 \\
    3 & 4 \\
  \end{pmatrix}
  \begin{pmatrix}
    5 & 6 \\
    7 & 8 \\
  \end{pmatrix}
  =
  \begin{pmatrix}
    19 & 22 \\
    43 & 50 \\
  \end{pmatrix}
\end{equation*}
であり,実行後の変数メモリで積の値が格納される
\verb|matrix3|に該当する\verb|0x10004020|からはじまる16バイトの
領域において
\begin{description}
  \item[第(1, 1)要素(\texttt{a[0][0]})] $\verb|13|_{(16)} = 19$  
  \item[第(1, 2)要素(\texttt{a[0][1]})] $\verb|16|_{(16)} = 22$  
  \item[第(2, 1)要素(\texttt{a[1][0]})] $\verb|2b|_{(16)} = 43$  
  \item[第(2, 2)要素(\texttt{a[1][1]})] $\verb|32|_{(16)} = 50$  
\end{description}
となっており,正しく行列の積が計算されたことが確認できる.

\paragraph*{実行時の命令・サイクル数}
\begin{Verbatim}[numbers=none, frame=single,
fontsize=\small, baselinestretch=0.8]
*** multicycle statistics ***
*** total inst. count:         1097
*** total cycle count:         4450
***               IPC:   0.247 (inst/cycle)
***               CPI:   4.054 (cycle/inst)
\end{Verbatim}

\paragraph*{使用した変数の情報}
\begin{Verbatim}[numbers=none, frame=single,
  fontsize=\small, baselinestretch=0.8]
# ------ [ Symbols ] ------
# 	symbol_0 	size: 16 	offset:     0(0) 	[matrix1]
# 	symbol_1 	size: 16 	offset:  0x10(16) 	[matrix2]
# 	symbol_2 	size: 16 	offset:  0x20(32) 	[matrix3]
# 	symbol_3 	size: 4 	offset:  0x30(48) 	[i]
# 	symbol_4 	size: 4 	offset:  0x34(52) 	[j]
# 	symbol_5 	size: 4 	offset:  0x38(56) 	[k]  
\end{Verbatim}

\paragraph*{実行後の変数メモリ}
出力されたメモリ領域が大きいため,省略した冒頭の一部を示す.

\begin{Verbatim}[numbers=none, frame=single,
  fontsize=\small, baselinestretch=0.8]
# tag=256 index=4
10004000: 00000001 00000002 00000003 00000004
10004010: 00000005 00000006 00000007 00000008
10004020: 00000013 00000016 0000002b 00000032
10004030: 00000002 00000002 00000002 00000000
\end{Verbatim}


%--------------------------------------------------------------------%
\section{考察}
今回のコンパイラプログラムの作成においては,
処理の追加実装や改良が行い易いように
抽象構文木を再帰的に探索してコード生成ができるノードの構成に
なるよう作成を行った.これにより
変数宣言以降の処理文集合について,
内部にさらに文が続くループ文や分岐処理文の実装が比較的容易に行えたため,
文集合だけでなく,特に他の処理でも多く利用される算術式などは
個々の演算子のノードのみを構文木に残すのではなく,
算術式という括りのノードを残すことによって,これを用いる
様々な文法規則への応用や機能の実装と拡張がより容易に行えると考えられる.

また,プログラムの実装の観点において,先述の様に再帰的な解析が可能な抽象構文木の作成と
その構造やノードに合わせたコード生成の関数などを用いる方針によって
プログラムの実装と機能追加は容易になったが,その反面再帰的処理の
流れを変えてしまうとコードの生成に支障をきたす箇所が多く,
最適化や算術式の計算結果の保持などに関する処理は多くの分岐処理
を加えることが必要となった.そのため,単に構文木を探索して都度コードの生成を行う
だけでなく,解析の後にコード生成における処理を選択して切り替える,
変数の記号表の様に全体の構造を抽象化したデータを作成するといった様々な
実装の方針を立てることがさらなる拡張や修正に
おける柔軟性を上げるために重要であると考えられる.

%--------------------------------------------------------------------%

% Verbatim environment
% プリアンブルで \usepackage{fancyvrb} が必要.
%   - numbers           行番号を表示.left なら左に表示.
%   - xleftmargin       枠の左の余白.行番号表示用に余白を与えたい.
%   - numbersep         行番号と枠の間隙 (gap).デフォルトは 12 pt.
%   - fontsize          フォントサイズ指定
%   - baselinestretch   行間の大きさを比率で指定.デフォルトは 1.0.
% \begin{Verbatim}[numbers=left, xleftmargin=10mm, numbersep=6pt,
%                     fontsize=\small, baselinestretch=0.8]
% #include <stdio.h>

% int main()
% {
%     char s[4] = {'A', 'B', 'C', '\0'};

%     printf("s = %s\n", s);

%     return 0;
% }
% \end{Verbatim}

%--------------------------------------------------------------------%
% 参考文献
%   以下は,書き方の例である.実際に,参考にした書籍等を見て書くこと.
%   本文で引用する際は,\cite{book:algodata}などとすればよい.
% \begin{thebibliography}{99}
%   \bibitem{book:algodata} 平田富雄,アルゴリズムとデータ構造,森北出版,1990.
%   \bibitem{book:label2} 著者名,書名,出版社,発行年.
%   \bibitem{www:label3} WWWページタイトル,URL,アクセス日.
% \end{thebibliography}

%--------------------------------------------------------------------%
%% 本文はここより上に書く(\begin{document}~\end{document}が本文である)
\end{document}
